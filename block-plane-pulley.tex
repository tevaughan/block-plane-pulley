
\documentclass[twocolumn]{article}

\usepackage{amsmath}
\usepackage{graphicx}
\usepackage{times}

\begin{document}

\section{Problem}

The flat bottom of a block B1 of mass~$m$ lies on a plane inclined at an
angle~$\theta$ to the horizontal. The coefficient of kinetic friction between
the plane and the block is $\mu$.  A massless, taut, inelastic cable runs from
B1 toward the top of the plane, around a pulley, and straight down to another
block B2, which also has mass~$m$.  The pulley's disc has infinite static
friction against the cable but requires no energy to separate from the cable as
it moves. The pulley's bearing is frictionless. The disc of the pulley has
uniform density, total mass $M$, and radius $r$. Find the acceleration $a$ of
the blocks; express $a$ in terms of $\mu$, $m$, $M$, $\theta$, and the
acceleration $g$ of gravity.

\section{Solution}

Each of B1, the pulley, and B2 must undergo the same acceleration because they
move as a rigid system: B1 and B2 maintain a constant distance of separation
along the length of the cable, and the cable does not slip along the surface of
the pulley.  The net force on each of B1, the pulley, and B2, when divided by
the relevant mass, must yield the same acceleration.

\subsection{Forces at B1}

Suppose that the tension in the cable attached to B1 is $T_1$. There are three
forces, parallel to the plane, acting on B1:
\begin{enumerate}
      \item a force of magnitude $T_1$ toward the pulley,
      \item the frictional force $\mu m g \cos(\theta)$ away from the pulley,
         and
      \item the plane-parallel component $m g \sin(\theta)$ of B1's weight.
\end{enumerate}
The acceleration, as derived from forces acting on B1, is
\begin{eqnarray}
   \nonumber
   a &=& \frac{T_1 - \mu m g \cos(\theta) - m g \sin(\theta)}{m}\\
   a &=& \frac{T_1}{m} - [\mu \cos(\theta) + \sin(\theta)] g.
   \label{eq:B1}
\end{eqnarray}

\subsection{Forces at the Pulley}

Suppose that the tension in the cable attached to B2 is $T_2$. There are two
forces, along the direction of the cable, acting on the pulley:
\begin{enumerate}
   \item a force of magnitude $T_1$ toward B1 and
   \item a force of magnitude $T_2$ toward B2.
\end{enumerate}
There are two corresponding torques:
\begin{enumerate}
   \item a torque $r T_1$ tending to spin the pulley so that B1 accelerates
      down the plane and
   \item a torque $r T_2$ tending to spin the pulley so that B1 accelerates up
      the plane.
\end{enumerate}
The angular acceleration, derived from torques acting on the pulley, is
\begin{eqnarray}
   \nonumber
   \alpha &=& \frac{r T_2 - r T_2}{\tfrac{1}{2} M r^2}\\
   \alpha &=& \frac{2 \: [T_2 - T_1]}{M r}.
\end{eqnarray}
The acceleration, as derived from forces acting on the pulley, is
\begin{eqnarray}
   \nonumber
   a &=& r \alpha\\
   a &=& \frac{2 \: [T_2 - T_1]}{M}.
   \label{eq:pulley}
\end{eqnarray}

\subsection{Forces at B2}

There are two forces acting on B2:
\begin{enumerate}
   \item a force of magnitude $T_2$ toward the pulley and
   \item B2's weight $mg$ away from the pulley.
\end{enumerate}
The acceleration, as derived from forces acting on B2, is
\begin{eqnarray}
   \nonumber
   a &=& \frac{mg - T_2}{m}\\
   a &=& g - \frac{T_2}{m}.
   \label{eq:B2}
\end{eqnarray}

\subsection{Tension Acting on B2}

We can, by combining Equation~\ref{eq:pulley} and Equation~\ref{eq:B2}, solve
for the tension $T_2$ acting on B2.
\begin{eqnarray}
   \nonumber
   \frac{2 \: [T_2 - T_1]}{M} &=& g - \frac{T_2}{m}\\
   \nonumber
   \frac{2 \: T_2}{M} + \frac{T_2}{m} &=& g + \frac{2 \: T_1}{M}\\
   \nonumber
   \left[ \frac{2}{M} + \frac{1}{m} \right] T_2 &=& g + \frac{2 \: T_1}{M}\\
   \nonumber
   \left[ 2 + \frac{M}{m} \right] T_2 &=& M g + 2 \: T_1\\
   T_2 &=& \frac{M g + 2 \: T_1}{2 + \tfrac{M}{m}}
   \label{eq:T2}
\end{eqnarray}
Here we have expressed the result in terms of the ratio $\tfrac{M}{m}$, which,
along with $\mu$ and $\theta$, is one of the three fundamental parameters that
determine the solution.

\subsection{Tension Acting on B1}

Now, combining Equation~\ref{eq:B1}, Equation~\ref{eq:B2}, and
Equation~\ref{eq:T2}, we can solve for the tension $T_1$ acting on B1.
\begin{eqnarray}
   \nonumber
   g - \frac{T_2}{m} &=& \frac{T_1}{m} - [\mu \cos(\theta) + \sin(\theta)] g\\
   \nonumber
   \frac{T_1 + T_2}{mg} &=& 1 + \mu \cos(\theta) + \sin(\theta)\\
   \nonumber
   \frac{T_1}{mg} \left[ \frac{4 + \tfrac{M}{m}}{2 + \tfrac{M}{m}} \right] &=&
   \frac{2}{2 + \tfrac{M}{m}} + \mu \cos(\theta) + \sin(\theta)
\end{eqnarray}
\begin{equation}
   T_1 = \left[ \frac{2 + [\mu \cos(\theta) + \sin(\theta)][2 +
   \tfrac{M}{m}]}{4 + \tfrac{M}{m}} \right] mg
   \label{eq:T1}
\end{equation}

\subsection{Acceleration of Blocks}

Finally, combining Equation~\ref{eq:T1} and Equation~\ref{eq:B1}, we can solve
for the acceleration.
\begin{small}
\begin{equation*}
   \frac{a}{g} = \frac{2 + [\mu \cos(\theta) + \sin(\theta)][2 +
   \tfrac{M}{m}]}{4 + \tfrac{M}{m}} - [\mu \cos(\theta) + \sin(\theta)]
\end{equation*}
\end{small}
\begin{equation}
   \frac{a}{g} = \left[ \frac{2}{4 + \tfrac{M}{m}} \right] [1 - \mu
   \cos(\theta) - \sin(\theta)]
\end{equation}
Again, we see that the acceleration depends on three parameters:
$\tfrac{M}{m}$, $\mu$, and $\theta$. The form of the solution shows that the
acceleration can range from a minimum of zero to a maximum of $\tfrac{g}{2}$.

\subsubsection{Minimum Acceleration}

The minimum ($a = 0$) obtains when either of the following is true:
\begin{itemize}
   \item $\tfrac{M}{m}$ approaches infinity, or
   \item $\mu \cos(\theta) + \sin(\theta) \geq 1$.
\end{itemize}
Considering the second case, we see that, for a given $\theta$, when $\mu$ is
larger than $\mu_\text{max} = \sec(\theta) - \tan(\theta)$, the kinetic
friction is large enough to decelerate the system and halt motion.
\begin{figure}
   \includegraphics[width=\columnwidth]{max-mu.png}
   \label{fig:mu-max}
\end{figure}
Figure~\ref{fig:mu-max} shows the how $\mu_\text{max}$ varies with $\theta$.
For small $\theta$, $\mu$ can approach unity, and the system will still
accelerate. For values of $\theta$ approaching 90~degrees, however, only a
small $\mu$ allows the system to accelerate.

\subsubsection{Maximum Acceleration}

The maximum ($a = \tfrac{g}{2}$) obtains when all of the following are true at
the same time:
\begin{itemize}
   \item $\tfrac{M}{m}$ approaches zero;
   \item $\mu = 0$; and
   \item $\theta = 0$.
\end{itemize}

\subsubsection{Summary}

\begin{figure*}
   \includegraphics[width=\textwidth]{accel-001.png}
   \label{fig:accel-001}
\end{figure*}

\begin{figure*}
   \includegraphics[width=\textwidth]{accel-010.png}
   \label{fig:accel-010}
\end{figure*}

\begin{figure*}
   \includegraphics[width=\textwidth]{accel-100.png}
   \label{fig:accel-100}
\end{figure*}

Figure~\ref{fig:accel-001},~\ref{fig:accel-010}, and~\ref{fig:accel-100}
summarize the result.

\end{document}

