
\newcommand{\doctitle}{How To Solve a Physics Problem}

\documentclass[twocolumn]{article}

\usepackage{amsmath}
\usepackage{amsfonts}  % for example, \mathbb

% for formatting figure captions
\usepackage[margin=10pt,font={sf},labelfont=bf]{caption}

\usepackage{enumitem}  % to allow for a compact list with [noitemsep]
\usepackage{fancyhdr}
\usepackage{graphicx}
\usepackage{lastpage}
\usepackage{tikz}
\usepackage{times}
\usepackage{vmargin}
\usepackage{xcolor}

% This must be the last package.
\usepackage[colorlinks=true,citecolor=blue,hyperfootnotes=false]{hyperref}

% vmargin setup
\setpapersize{USletter}
\setmarginsrb%
{0.375in}%           left
{0.375in}%           top
{0.375in}%           right
{0.5in}%             bottom
{2\baselineskip}%    headheight
{2\baselineskip}%    headsep
{3\baselineskip}%    footheight
{4\baselineskip}%    footskip

% mydate macro
\newcommand{\mydate}{%
   \number\year\space%
   \ifcase\month\or%
      Jan\or\ Feb\or\ Mar\or\ Apr\or\ May\or\ Jun\or%
      Jul\or\ Aug\or\ Sep\or\ Oct\or\ Nov\or\ Dec
   \fi\space%
   \number\day%
}

% fancyhdr settings
\pagestyle{fancy}
\lhead{\sffamily\textbf{\doctitle}}
\chead{}
\rhead{\sffamily \thepage~of~\pageref{LastPage}}
\renewcommand{\headrulewidth}{1pt}
\renewcommand{\footrulewidth}{1pt}
\lfoot{%
   \footnotesize\sffamily
   \begin{minipage}{0.95\textwidth}
   Copyright\ \copyright\ \ 2017\ \ Thomas E.\ Vaughan.
   PDF image generated on \mydate.
   Permission is granted to copy, distribute and/or modify this document under
   the terms of the GNU Free Documentation License, Version 1.3 or any later
   version published by the Free Software Foundation; with no Invariant
   Sections, no Front-Cover Texts, and no Back-Cover Texts.  A copy of the
   license is included in the section entitled ``GNU Free Documentation
   License''.
   \end{minipage}%
}
\cfoot{}
\rfoot{%
   \begin{minipage}{0.05\textwidth}
   \begin{flushright}
   \includegraphics[width=0.85\textwidth]{logo}
   \end{flushright}
   \end{minipage}%
}

\begin{document}

\thispagestyle{fancy}

\begin{abstract}

   An example of how to solve a high-school-level theoretical physics problem
   is presented.  Involving a block sliding on an inclined plane, a pulley with
   rotational inertia, and a second block serving as a counterweight, the
   problem is moderately complex.  No calculus is required.  The example shows
   the approach in which no numerical value for any parameter is assumed in the
   solution.  The solution is obtained for the general case.  The specific,
   limiting cases are explored.  Finally, a summary of the parameter space is
   presented in a collection of graphs.  This approach allows one to grasp all
   of the modes of behavior that various individual systems of the same type
   can display and how the different behaviors relate to the parameters
   defining the type.

\end{abstract}

\section{Introduction}

Many a physics problem aimed at the high-school student is stated so that every
element of the problem has a specified numerical value.  There are good reasons
to state a problem with numerical specificity and to ask for a numerical
solution.  Numerical specificity allows for the easy multiplication of
different particular problems of the same general type.  This is convenient for
the teacher.  Further, the student should be able not only to demonstrate
conceptual mastery of the subject but also to calculate an accurate answer in a
specific case.  Numerical practice is good for the student.  Also, experimental
physics is inherently numerical. Almost any problem involving the statistical
analysis of experimental data is likely to require numerical particulars. For
all of these reaons, one might argue that, as a matter of good practice, the
majority of problems should be stated with numerical specificity.

Yet the ultimate purpose of giving to a student a theoretical physics problem
(not a statistics problem) is to allow him to demonstrate conceptual mastery of
some aspect of theoretical physics.  However useful numerical calculation may
be as a skill, it can be learned and demonstrated elsewhere, entirely apart
from a curriculum on theoretical physics.  The conceptual mastery necessary for
the demonstration of ability in theoretical physics is inherently abstracted
away from the numerical particulars.  The teacher of physics ought therefore to
emphasize a conceptual method of solution, even for a problem that happens to
specify numerical particulars.  That is, the teacher might well require that
the student solve every problem in a perfectly general way and then, perhaps as
a final step, make numerical substitutions and arithmetic calculations. At
least on occasion, the teacher should ask the student to reflect on the general
solution in order to answer various conceptual questions, such as whether the
dimensions make sense, whether there might be, for a given quantity, a critical
value beyond which the behavior of the system changes qualitatively, etc.

The problem considered below has no numerical specificity, but the behavior of
the physical system is, at the end of the paper, analyzed across a wide range
of numerical specifics.

\section{Problem}

The flat bottom of a block B1 of mass~$m$ lies on a plane inclined at an
angle~$\theta$ to the horizontal. See Figure~\ref{fig:diagram}. The coefficient
of kinetic friction between the plane and the block is $\mu$.  A massless,
taut, inelastic cable runs from B1 toward the top of the plane, around a
pulley, and straight down to another block B2, which also has mass~$m$.  The
pulley's disc has infinite static friction against the cable but requires no
energy to separate from the cable as it moves. The pulley's bearing is
frictionless. The disc of the pulley has uniform density, total mass $M$, and
radius $r$. Find the acceleration $a$ of the blocks; express $a$ in terms of
$\mu$, $\tfrac{M}{m}$, $\theta$, and the acceleration $g$ of gravity.

\section{Solution}

Each of B1, the pulley, and B2 must undergo the same acceleration because they
move as a rigid system: B1 and B2 maintain a constant distance of separation
along the length of the cable, and the cable does not slip along the surface of
the pulley.  The net force on each of B1, the pulley, and B2, when divided by
the relevant mass, must yield the same acceleration.

\subsection{Forces at B1}

\begin{figure}
   \begin{center}
      \includegraphics[width=0.95\columnwidth]{diagram}
   \end{center}
   \caption{Diagram of system. Some of the system parameters ($\theta$, $m$,
   and $M$) are indicated. The coefficient $\mu$ of friction corresponds to the
   surface on which block B1 sits. Although the radius $r$ of the pulley's disc
   is indicated, $r$ is eliminated from all of the relevant expressions and is
   therefore not a key parameter of the system's behavior.}
   \label{fig:diagram}
\end{figure}

Suppose that the tension in the cable attached to B1 is $T_1$. There are three
forces, parallel to the plane, acting on B1:
\begin{enumerate}
      \item toward the pulley, a tensile force of magnitude $T_1$,
      \item away from the pulley, the frictional force $\mu m g \cos(\theta)$
         due to the plane-perpendicular component of B1's weight, and
      \item away from the pulley, the plane-parallel component $m g
         \sin(\theta)$ of B1's weight.
\end{enumerate}
The acceleration, as derived from forces acting on B1, is
\begin{eqnarray}
   \nonumber
   a &=& \frac{T_1 - \mu m g \cos(\theta) - m g \sin(\theta)}{m}\\
   a &=& \tfrac{T_1}{m} - [\mu \cos(\theta) + \sin(\theta)] g.
   \label{eq:B1}
\end{eqnarray}

\subsection{Forces at the Pulley}

Suppose that the tension in the cable attached to B2 is $T_2$. There are two
forces, along the direction of the cable, acting on the pulley:
\begin{enumerate}[noitemsep]
   \item toward B1, a tensile force of magnitude $T_1$ and
   \item toward B2, a tensile force of magnitude $T_2$.
\end{enumerate}
There are two corresponding torques:
\begin{enumerate}
   \item a torque $\tau_1 = r T_1$ tending to spin the pulley so that B1
      accelerates down the plane and
   \item a torque $\tau_2 = r T_2$ tending to spin the pulley so that B1
      accelerates up the plane.
\end{enumerate}
The angular acceleration, derived from torques acting on the pulley, is
\begin{equation}
   \alpha = \frac{\tau_2 - \tau_1}{I} = \frac{2 \: [T_2 - T_1]}{M r},
\end{equation}
where $I = \tfrac{1}{2} M r^2$ is the rotational inertia of the pulley's disc.
The acceleration, as derived from forces acting on the pulley, is
\begin{equation}
   a = r \alpha = \tfrac{2}{M} [T_2 - T_1].
   \label{eq:pulley}
\end{equation}

\subsection{Forces at B2}

There are two forces acting on B2:
\begin{enumerate}[noitemsep]
   \item toward the pulley, a tensile force of magnitude $T_2$ and
   \item away from the pulley, B2's weight $mg$.
\end{enumerate}
The acceleration, as derived from forces acting on B2, is
\begin{equation}
   a = \frac{mg - T_2}{m} = g - \tfrac{T_2}{m}.
   \label{eq:B2}
\end{equation}

\subsection{Tension Acting on B2}

We can, by combining Equation~\ref{eq:pulley} and Equation~\ref{eq:B2}, solve
for the tension $T_2$ acting on B2.
\begin{eqnarray}
   \nonumber
   \tfrac{2}{M} [T_2 - T_1] &=& g - \tfrac{T_2}{m}\\
   \nonumber
   2 \: [T_2 - T_1] &=& \tfrac{M}{m} [mg - T_2]\\
   \nonumber
   \left[2 + \tfrac{M}{m}\right] T_2 &=& \left[\tfrac{M}{m}\right] mg + 2 \: T_1\\
   T_2 &=& \left[\frac{\tfrac{M}{m}}{2 + \tfrac{M}{m}}\right] mg +
           \left[\frac{2}{2 + \tfrac{M}{m}}\right] T_1
   \label{eq:T2}
\end{eqnarray}
Here we have expressed the result in terms of the ratio $\tfrac{M}{m}$, which,
along with $\mu$ and $\theta$, is one of the three fundamental parameters that
determine the solution.

\subsection{Tension Acting on B1}

Now, combining Equation~\ref{eq:B1}, Equation~\ref{eq:B2}, and
Equation~\ref{eq:T2}, we can solve for the tension $T_1$ acting on B1.
\begin{eqnarray}
   \nonumber
   g - \tfrac{T_2}{m} &=& \tfrac{T_1}{m} - [\mu \cos(\theta) + \sin(\theta)]
                          g\\
   \nonumber
   \frac{T_1 + T_2}{mg} &=& 1 + \mu \cos(\theta) + \sin(\theta)\\
   \nonumber
   \frac{T_1}{mg} \left[ \frac{4 + \tfrac{M}{m}}{2 + \tfrac{M}{m}} \right] &=&
   \frac{2}{2 + \tfrac{M}{m}} + \mu \cos(\theta) + \sin(\theta)
\end{eqnarray}
\begin{equation}
   T_1 = \left[ \frac{2 + [\mu \cos(\theta) + \sin(\theta)]\left[2 +
   \tfrac{M}{m}\right]}{4 + \tfrac{M}{m}} \right] mg
   \label{eq:T1}
\end{equation}

\subsection{Acceleration of Blocks}

Finally, combining Equation~\ref{eq:T1} and Equation~\ref{eq:B1}, we can solve
for the acceleration.
\begin{eqnarray}
   \nonumber
   \tfrac{a}{g} &=& \frac{2 + [\mu \cos(\theta) + \sin(\theta)][2 +
                     \tfrac{M}{m}]}{4 + \tfrac{M}{m}} - [\mu \cos(\theta) +
                     \sin(\theta)]\\
   \tfrac{a}{g} &=& \left[ \frac{2}{4 + \tfrac{M}{m}} \right] [1 - \mu
                    \cos(\theta) - \sin(\theta)]
\end{eqnarray}
Again, we see that the acceleration depends on three parameters:
$\tfrac{M}{m}$, $\mu$, and $\theta$. The form of the solution shows that the
acceleration can range from a minimum of zero to a maximum of $\tfrac{g}{2}$.

\subsubsection{Minimum Acceleration}

The minimum acceleration ($a = 0$) obtains when either of the following is
true:
\begin{itemize}[noitemsep]
   \item $\tfrac{M}{m}$ approaches infinity, or
   \item $\mu \cos(\theta) + \sin(\theta) = 1$.
\end{itemize}
Considering the second case, we see that, for a given $\theta$, when $\mu$ is
larger than the critical value $\mu_\text{crit} = \sec(\theta) - \tan(\theta)$,
the kinetic friction is large enough to decelerate the system. In such a case,
the deceleration would stop when the system comes to rest, for at that moment
the kinetic frictional force would cease to exist.
\begin{figure}
   \begin{center}
      \includegraphics[width=0.95\columnwidth]{crit-mu.png}
   \end{center}
   \caption{Plotted against the angle of inclination of the plane, the critical
   value $\mu_\text{crit}$ of the coefficient of kinetic friction. For any
   coefficient above this limit, a moving system decelerates, and a stationary
   system remains stationary.}
   \label{fig:mu-crit}
\end{figure}
Figure~\ref{fig:mu-crit} shows the how $\mu_\text{crit}$ varies with $\theta$.
For small $\theta$, $\mu$ can approach unity, and the system will still
accelerate. For values of $\theta$ approaching 90~degrees, however, only a
small $\mu$ allows the system to accelerate.

\subsubsection{Maximum Acceleration}

The maximum acceleration ($a = \tfrac{g}{2}$) obtains when all of the following
are true at the same time:
\begin{itemize}[noitemsep]
   \item $\tfrac{M}{m} = 0$;
   \item $\mu = 0$; and
   \item $\theta = 0$.
\end{itemize}

\subsubsection{Maximum Deceleration}

The maximum deceleration ($a = -\tfrac{g}{2\sqrt{2}}$) obtains when all of the
following are true at the same time:
\begin{itemize}[noitemsep]
   \item $\tfrac{M}{m} = 0$;
   \item $\mu = 1$; and
   \item $\theta = \tfrac{\pi}{4}$.
\end{itemize}

\subsubsection{Summary}

\begin{figure}
   \begin{center}
      \includegraphics[width=0.9\columnwidth]{accel-001.png}
      \includegraphics[width=0.9\columnwidth]{accel-010.png}
      \includegraphics[width=0.9\columnwidth]{accel-100.png}
   \end{center}
   \caption{Contours of constant acceleration in $\theta$-$\mu$ plane for each
            of a small, a mid-range, and a large value of $\tfrac{M}{m}$. The
            top plot corresponds to $\tfrac{M}{m} = 0.1$; the middle, to
            $\tfrac{M}{m} = 1.0$; and the bottom, to $\tfrac{M}{m} = 10.0$.
            Each contour is labeled with the corresponding acceleration as a
            fraction of $g$.}
   \label{fig:accel}
\end{figure}

Figure~\ref{fig:accel} summarizes the result. Notice that the curve for zero
acceleration in each of the three plots is the same as the curve in
Figure~\ref{fig:mu-crit}. Each curve for negative acceleration applies only to
a system that is initially moving and indicates frictional deceleration. For
positive $\mu$, friction is present while the system is moving. If the system
be initially pushed into a state of motion, then there are three modes of
behavior, and which behavior is expressed depends on the value of $\mu$.
\begin{enumerate}
   \item Friction dominates the behavior, by slowing the system to a halt, when
      $\mu > \mu_\text{crit} = \sec(\theta) - \tan(\theta)$.
   \item When $\mu = \mu_\text{crit}$, the system will continue to move at the
      initial velocity.
   \item Otherwise, when $\mu < \mu_\text{crit}$, the system will accelerate
      toward velocities greater than the initial velocity.
\end{enumerate}

\section{Conclusion}

Without considering any arbitrary, numerical particulars, we have gained a
thorough understanding of every particular problem corresponding to
Figure~\ref{fig:diagram}. This understanding can be achieved by the ordinary
student and does not require the production of graphs such as the contour plots
above. (Those plots do aid understanding, but all of the basic properties of
the system can be determined through ordinary algebra and perhaps a couple of
simple plots.) Whenever a student applies this same approach to any problem, he
does what is proper to theoretical physics.

The full source code to this article, including the text and the source code
for the figures, is available here:
\url{https://github.com/tevaughan/block-plane-pulley}.

\newpage

\input{fdl-1.3}

\end{document}

